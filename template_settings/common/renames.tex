%%% Внесите свои данные - Input your data
%%
%%
\newcommand{\Author}{И.И.\,Хамидуллин} % И.О. Фамилия автора 
\newcommand{\AuthorFull}{Хамидуллин Ильсаф Ильназович} % Фамилия Имя Отчество автора
\newcommand{\AuthorFullDat}{Хамидуллину Ильсафу Ильназовичу} % Фамилия Имя Отчество автора в дательном падеже (Кому? Студенту...)
\newcommand{\AuthorFullVin}{Хамидуллина Ильсафа Ильназовича} % в винительном падеже (Кого? что?  Програмиста ...)
\newcommand{\AuthorPhone}{+7-986-719-57-30} % номер телефорна автора для оперативной связи  
\newcommand{\Supervisor}{Ф.А.\,Новиков} % И. О. Фамилия научного руководителя
\newcommand{\SupervisorFull}{Новиков Федор Александрович} % Фамилия Имя Отчество научного руководителя
\newcommand{\SupervisorVin}{Ф.А.\,Новикову} % И. О. Фамилия научного руководителя  в винительном падеже (Кого? что? Руководителя ...)
\newcommand{\SupervisorJob}{д.т.н.} %
\newcommand{\SupervisorJobVin}{д.т.н} % в винительном падеже (Кого? что?  Програмиста ...)
\newcommand{\SupervisorDegree}{д.т.н} %
\newcommand{\SupervisorTitle}{профессор ВШПМиВФ} % 
%%
%%
%Руководитель, утверждающий задание
\newcommand{\Head}{К.Н.\,Козлов} % И. О. Фамилия руководителя подразделения (руководителя ОП)
\newcommand{\HeadDegree}{Руководитель образовательной программы}% Только должность:   
%Руководитель ОП % <- ПРАВИЛЬНЫЙ ВАРИАНТ ДЛЯ 09.03.03 и 09.04.03
%Заведующий % кафедрой
%Директор % Высшей школы
%Зам. директора
\newcommand{\HeadDep}{} % заменить на краткую аббревиатуру подразделения или ОСТАВИТЬ ПУСТЫМ, если утверждает руководитель ОП

%%% Руководитель, принимающий заявление
\newcommand{\HeadAp}{К.Н.\,Козлов} % И. О. Фамилия руководителя подразделения (руководителя ОП)
\newcommand{\HeadApDegree}{Руководитель образовательной программы «Прикладная математика и информатика»}% Только должность:   
%Руководитель ОП 
%Заведующий кафедрой
%Директор Высшей школы
\newcommand{\HeadApDep}{} % заменить на краткую аббревиатуру подразделения или оставить пустым, если утверждает руководитель ОП
%%% Консультант по нормоконтролю
\newcommand{\ConsultantNorm}{Л.А.\,Арефьева} % И. О. Фамилия консультанта по нормоконтролю. ТОЛЬКО из числа ППС!
\newcommand{\ConsultantNormDegree}{должность, степень} %   
%%% Первый консультант
\newcommand{\ConsultantExtraFull}{Иванов Денис Юрьевич} % Фамилия Имя Отчетство дополнительного консультанта 
\newcommand{\ConsultantExtra}{Д.Ю.\,Иванов} % И. О. Фамилия дополнительного консультанта 
\newcommand{\ConsultantExtraDegree}{должность, степень} % 
\newcommand{\ConsultantExtraVin}{И.О.\,Фамилию} % И. О. Фамилия дополнительного консультанта в винительном падеже (Кого? что? Руководителя ...)
\newcommand{\ConsultantExtraDegreeVin}{должность, степень} %  в винительном падеже (Кого? что? Руководителя ...)
%%% Второй консультант
\newcommand{\ConsultantExtraTwoFull}{Фамилия Имя Отчетство} % Фамилия Имя Отчетство дополнительного консультанта 
\newcommand{\ConsultantExtraTwo}{И.О.\,Фамилия} % И. О. Фамилия дополнительного консультанта 
\newcommand{\ConsultantExtraTwoDegree}{должность, степень} % 
\newcommand{\ConsultantExtraTwoVin}{И.О.\,Фамилию} % И. О. Фамилия дополнительного консультанта в винительном падеже (Кого? что? Руководителя ...)
\newcommand{\ConsultantExtraTwoDegreeVin}{должность, степень} %  в винительном падеже (Кого? что? Руководителя ...)
\newcommand{\Reviewer}{И.О.\,Фамилия} % И. О. Фамилия резензента. Обязателен только для магистров.
\newcommand{\ReviewerDegree}{должность, степень} % 
%%
%%
\renewcommand{\thesisTitle}{Автоматическая генерация конфигураций элементов инфраструктуры программных систем для работы с большими данными}
\newcommand{\thesisDegree}{работа бакалавра}% дипломный проект, дипломная работа, магистерская диссертация %c 2020
\newcommand{\thesisTitleEn}{Automatic generation of configurations for infrastructure elements of software systems for big data processing} %2020
\newcommand{\thesisDeadline}{июнь 2025 г} %Последний день преддипломной практики согласно учебному плану.
\newcommand{\thesisStartDate}{03.02.2025}
\newcommand{\thesisYear}{2025} % заменить на год защиты
\newcommand{\approveYear}{2025} % <- НЕ МЕНЯТЬ, ТОЛЬКО С 2030го :)
%%
%%
\newcommand{\group}{5030102/10201} % заменить вместо N номер группы
\newcommand{\thesisSpecialtyCode}{01.03.02}% код направления подготовки
\newcommand{\thesisSpecialtyTitle}{Прикладная математика и информатика} % наименование направления/специальности
\newcommand{\thesisOPPostfix}{02} % последние цифры кода образовательной программы (после <<_>>)
\newcommand{\thesisOPTitle}{Системное программирование}% наименование образовательной программы
%%
%%
\newcommand{\institute}{
	Физико-механический институт
	%Институт компьютерных наук и~кибербезопасности
	%Гуманитарный институт
	%Инженерно-строительный институт
	%Институт биомедицинских систем и технологий
	%Институт металлургии, машиностроения и транспорта
	%Институт передовых производственных технологий
	%Институт прикладной математики и механики
	%Институт физики, нанотехнологий и телекоммуникаций
	%Институт физической культуры, спорта и туризма
	%Институт энергетики и транспортных систем
	%Институт промышленного менеджмента, экономики и торговли
}%
%%
%%




%%% Задание ключевых слов и аннотации
%%
%%
%% Ключевых слов от 3 до 5 слов или словосочетаний в именительном падеже именительном падеже множественного числа (или в единственном числе, если нет другой формы) по правилам русского языка!!!
%%
%%
\newcommand{\keywordsRu}{Автоматизация развертывания, большие данные, генерация конфигураций, декларативное описание, YAML, Docker Compose, Потоковая обработка в реальном времени, инфраструктура как код} % ВВЕДИТЕ ключевые слова по-русски
%%
%%
\newcommand{\keywordsEn}{Deployment automation, big data, configuration generation, declarative description, YAML, Docker, Docker Compose, real time stream processing, infrastructure as code}
% ВВЕДИТЕ ключевые слова по-английски
%%
%%
%% Реферат ОТ 1000 ДО 1500 знаков на русский или английский текст
%%
%Реферат должен содержать:
%- предмет, тему, цель ВКР;
%- метод или методологию проведения ВКР:
%- результаты ВКР:
%- область применения результатов ВКР;
%- выводы.

\newcommand{\abstractRu}{Дипломная работа посвящена актуальной проблеме автоматизации развертывания инфраструктуры для работы с большими данными. Целью работы является разработка программного инструмента dpd (Data Platform Deployer), способного на основе декларативного описания, предоставленного пользователем в формате YAML, генерировать полный набор конфигурационных файлов и скриптов для запуска комплексной платформы данных. Входное описание включает определение таких компонентов, как системы управления базами данных (PostgreSQL в качестве источника, ClickHouse в качестве аналитического хранилища), S3-совместимое объектное хранилище (Minio), брокер сообщений Apache Kafka с настроенными топиками и коннекторами Kafka Connect (включая Debezium для CDC и S3 Sink), а также инструмент бизнес-аналитики Apache Superset. Разработанный инструмент dpd автоматически формирует docker-compose.yml файлы для контейнеризации сервисов, скрипты их инициализации и обеспечивает согласованность настроек между всеми компонентами. Ключевыми преимуществами предлагаемого решения являются воспроизводимость конфигураций, значительное сокращение трудозатрат по сравнению с ручной настройкой, модульность для поддержки новых компонентов и обеспечение корректности взаимосвязей в развертываемой системе.} % ВВЕДИТЕ текст аннотации по-русски
%%
%%
\newcommand{\abstractEn}{In the given work the essence of the approach to creation of a dynamic information portal on the basis of use of open technologies Apache, MySQL and PHP is stated. The general concepts and classification of IT-systems of such class are given. The analysis of systems-prototypes is lead. The technology of creation of the specified class of information systems is investigated. Concrete program realization of a dynamic information portal on an example of a portal of the chosen subjects is developed...} % ВВЕДИТЕ текст аннотации по-английски


%%% РАЗДЕЛ ДЛЯ ОФОРМЛЕНИЯ ПРАКТИКИ
%Место прохождения практики
\newcommand{\PracticeType}{Отчет о прохождении % 
	%стационарной производственной (технологической (проектно-технологической)) %
	такой-то % тип и вид ЗАМЕНИТЬ
	практики}

\newcommand{\Workplace}{СПбПУ, ИКНК, ВШ ПИ} % TODO Rename this variable

% Даты начала/окончания
\newcommand{\PracticeStartDate}{%
	дд.мм.гггг%
	%	22.06.2020
}%
\newcommand{\PracticeEndDate}{%
	дд.мм.гггг%
	%	18.07.2020%
}%
%%

\newcommand{\School}{
	Название высшей школы
	%	Высшая школа программной инженерии 
}
\newcommand{\practiceTitle}{Тема практики}


%% ВНИМАНИЕ! Необходимо либо заменить текст аннотации (ключевых слов) на русском и английском, либо удалить там весь текст, иначе в свойства pdf-отчета по практике пойдет шаблонный текст.

%%% Не меняем дальнейшую часть - Do not modify the rest part
%%
%%
%%
%%
\ifnumequal{\value{docType}}{1}{% Если ВКР, то...
	\newcommand{\DocType}{Выпускная квалификационная работа}
	\newcommand{\pdfDocType}{\DocType~(\thesisDegree)} %задаём метаданные pdf файла
	\newcommand{\pdfTitle}{\thesisTitle}
}{% Иначе 
	\newcommand{\DocType}{\PracticeType}
	\newcommand{\pdfDocType}{\DocType} %задаём метаданные pdf файла
	\newcommand{\pdfTitle}{\practiceTitle}
}%
\newcommand{\HeadTitle}{\HeadDegree~\HeadDep}
\newcommand{\HeadApTitle}{\HeadApDegree~\HeadApDep}
\newcommand{\thesisOPCode}{\thesisSpecialtyCode\_\thesisOPPostfix}% код образовательной программы
\newcommand{\thesisSpecialtyCodeAndTitle}{\thesisSpecialtyCode~\thesisSpecialtyTitle}% Код и наименование направления/специальности
\newcommand{\thesisOPCodeAndTitle}{\thesisOPCode~\thesisOPTitle} % код и наименование образовательной программы
%%
%%
\hypersetup{%часть болка hypesetup в style
	pdftitle={\pdfTitle},    % Заголовок pdf-файла
	pdfauthor={\AuthorFull},    % Автор
	pdfsubject={\pdfDocType. Шифр и наименование направления подготовки: \thesisSpecialtyCodeAndTitle. \abstractRu},      % Тема
	pdfcreator={LaTeX, SPbPU-student-thesis-template},     % Приложение-создатель
	%		pdfproducer={},  % Производитель, Производитель PDF % будет выставлена автоматически
	pdfkeywords={\keywordsRu}
}
%%
%%
%% вспомогательные команды
\newcommand{\firef}[1]{рис.\ref{#1}} %figure reference
\newcommand{\taref}[1]{табл.\ref{#1}}	%table reference
%%
%%
%% Архивный вариант задания ключевых слов, аннотации и благодарностей 
% Too hard to export data from the environment to pdf-info
% https://tex.stackexchange.com/questions/184503/collecting-contents-of-environment-and-store-them-for-later-retrieval
%заменить NewEnviron на newenvironment для распознавания команды в TexStudio
%\NewEnviron{keywordsRu}{\noindent\MakeUppercase{\BODY}}
%\NewEnviron{keywordsEn}{\noindent\MakeUppercase{\BODY}}
%\newenvironment{abstractRu}{}{}
%\newenvironment{abstractEn}{}{}
%\newenvironment{acknowledgementsRu}{\par{\normalfont \acknowledgements.}}{}
%\newenvironment{acknowledgementsEn}{\par{\normalfont \acknowledgementsENG.}}{}


%%% Переопределение именований %%% Не меняем - Do not modify
%\newcommand{\Ministry}{Минобрнауки России} 
\newcommand{\Ministry}{Министерство науки и высшего образования Российской~Федерации} %с 2020
\newcommand{\SPbPU}{Санкт-Петербургский политехнический университет Петра~Великого}
\newcommand{\SPbPUOfficialPrefix}{Федеральное государственное автономное образовательное учреждение высшего образования}
\newcommand{\SPbPUOfficialShort}{ФГАОУ~ВО~<<СПбПУ>>}
%% Пробел между И. О. не допускается.
\renewcommand{\alsoname}{см. также}
\renewcommand{\seename}{см.}
\renewcommand{\headtoname}{вх.}
\renewcommand{\ccname}{исх.}
\renewcommand{\enclname}{вкл.}
\renewcommand{\pagename}{Pages}
\renewcommand{\partname}{Часть}
\renewcommand{\abstractname}{\textbf{Аннотация}}
\newcommand{\abstractnameENG}{\textbf{Annotation}}
\newcommand{\keywords}{\textbf{Ключевые слова}}
\newcommand{\keywordsENG}{\textbf{Keywords}}
\newcommand{\acknowledgements}{\textbf{Благодарности}}
\newcommand{\acknowledgementsENG}{\textbf{Acknowledgements}}
\renewcommand{\contentsname}{Content} % 
%\renewcommand{\contentsname}{Содержание} % (ГОСТ Р 7.0.11-2011, 4)
%\renewcommand{\contentsname}{Оглавление} % (ГОСТ Р 7.0.11-2011, 4)
\renewcommand{\figurename}{Рис.} % Стиль СПбПУ
%\renewcommand{\figurename}{Рисунок} % (ГОСТ Р 7.0.11-2011, 5.3.9)
\renewcommand{\tablename}{Таблица} % (ГОСТ Р 7.0.11-2011, 5.3.10)
%\renewcommand{\indexname}{Предметный указатель}
\renewcommand{\listfigurename}{Список рисунков}
\renewcommand{\listtablename}{Список таблиц}
\renewcommand{\refname}{\fullbibtitle}
\renewcommand{\bibname}{\fullbibtitle}

\newcommand{\chapterEnTitle}{Сhapter title} % <- input the English title here (only once!) 
\newcommand{\chapterRuTitle}{Название главы}          % <- введите 
\newcommand{\sectionEnTitle}{Section title} %<- input subparagraph title in english
\newcommand{\sectionRuTitle}{Название подраздела} % <- введите название подраздела по-русски
\newcommand{\subsectionEnTitle}{Subsection title} % - input subsection title in english
\newcommand{\subsectionRuTitle}{Название параграфа} % <- введите название параграфа по-русски
\newcommand{\subsubsectionEnTitle}{Subsubsection title} % <- input subparagraph title in english
\newcommand{\subsubsectionRuTitle}{Название подпараграфа} % <- введите название подпараграфа по-русски