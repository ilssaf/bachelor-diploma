\chapter*{Словарь терминов}             % Заголовок
\addcontentsline{toc}{chapter}{Словарь терминов}  % Добавляем его в оглавление

\textbf{API (Application Programming Interface)} --- интерфейс программирования приложений; набор готовых классов, процедур, функций, структур и констант, предоставляемых приложением (библиотекой, сервисом) для использования во внешних программных продуктах.

\textbf{ANTLR (ANother Tool for Language Recognition)} --- генератор парсеров, который используется для создания компиляторов, интерпретаторов и других инструментов, работающих с языками программирования или формальными языками.

\textbf{Apache Kafka} --- распределённая платформа для обработки потоковых данных, используемая для построения конвейеров данных реального времени и потоковых приложений.

\textbf{Apache Superset} --- веб-приложение с открытым исходным кодом для исследования и визуализации данных.

\textbf{Arenadata} --- российская компания, разрабатывающая платформу для сбора, хранения и обработки больших данных на основе технологий с открытым исходным кодом.

\textbf{Docker} --- программная платформа для быстрой разработки, тестирования и развертывания приложений. Docker упаковывает программное обеспечение в стандартизированные блоки, называемые контейнерами, которые включают все необходимое для работы: библиотеки, системные инструменты, код и среду выполнения.

\textbf{Docker Compose} --- инструмент для определения и запуска многоконтейнерных приложений Docker. С помощью Compose используется YAML-файл для настройки служб приложения.

\textbf{DSL (Domain-Specific Language)} --- язык, специализированный для конкретной области применения. В контексте работы, это язык описания конфигурации платформы данных.

\textbf{ETL (Extract, Transform, Load)} --- извлечение, преобразование, загрузка; один из основных процессов в управлении хранилищами данных, который включает извлечение данных из внешних источников, их преобразование и очистку для соответствия нуждам бизнес-модели и загрузку в хранилище данных.

\textbf{IaC (Infrastructure as Code)} --- инфраструктура как код; подход к управлению и предоставлению компьютерных центров обработки данных через машиночитаемые файлы определений, а не через физическую конфигурацию оборудования или интерактивные инструменты настройки.

\textbf{JSON (JavaScript Object Notation)} --- текстовый формат обмена данными, основанный на JavaScript. Легко читаем людьми и легко обрабатывается компьютерами.

\textbf{Kafka Connect} --- фреймворк для надежной потоковой передачи данных между Apache Kafka и другими системами. Используется для создания коннекторов, которые перемещают большие наборы данных в Kafka и из Kafka.

\textbf{Minio} --- высокопроизводительное распределенное объектное хранилище, совместимое с Amazon S3 API.

\textbf{PostgreSQL} --- свободная объектно-реляционная система управления базами данных (СУБД).

\textbf{S3 (Simple Storage Service)} --- сервис простого хранения данных, изначально разработанный Amazon Web Services; стандарт де-факто для API объектных хранилищ.

\textbf{SQL (Structured Query Language)} --- язык структурированных запросов; декларативный язык программирования, применяемый для создания, модификации и управления данными в реляционной базе данных.

\textbf{Yandex Cloud} --- облачная платформа, предоставляемая компанией Яндекс, включающая различные сервисы, в том числе для работы с большими данными.

\textbf{YAML (YAML Ain't Markup Language)} --- рекурсивный акроним, «YAML — не язык разметки»; дружественный к человеку формат сериализации данных, часто используемый для конфигурационных файлов.

\textbf{Аналитика данных (Data Analytics)} --- процесс инспектирования, очистки, преобразования и моделирования данных с целью извлечения полезной информации, формирования выводов и поддержки принятия решений.

\textbf{Большие данные (Big Data)} --- совокупность подходов, инструментов и методов обработки структурированных и неструктурированных данных огромных объёмов и значительного многообразия для получения воспринимаемых человеком результатов.

\textbf{Брокер сообщений (Message Broker)} --- промежуточное программное обеспечение, которое преобразует сообщения из формального протокола обмена сообщениями отправителя в формальный протокол обмена сообщениями получателя.

\textbf{Визуализация данных (Data Visualization)} --- представление данных в графическом формате для облегчения их восприятия и анализа.

\textbf{Генератор конфигураций (Configuration Generator)} --- программный инструмент, который автоматически создает конфигурационные файлы для одной или нескольких систем на основе входных параметров или шаблонов.

\textbf{Декларативное описание (Declarative Description)} --- способ описания системы или процесса, при котором указывается *что* должно быть достигнуто, а не *как* это сделать.

\textbf{Интегрированная платформа данных (Integrated Data Platform)} --- комплексное решение, объединяющее различные инструменты и сервисы для сбора, хранения, обработки и анализа данных в единой среде.

\textbf{Инфраструктура (Infrastructure)} --- совокупность взаимосвязанных обслуживающих структур или объектов, составляющих и/или обеспечивающих основу функционирования системы. В ИТ это физическое и виртуальное оборудование, сети, операционные системы, хранилища данных.

\textbf{Коннектор (Connector)} --- в Kafka Connect, компонент, отвечающий за интеграцию с конкретным источником или приемником данных.

\textbf{Контейнеризация (Containerization)} --- метод виртуализации на уровне операционной системы, при котором приложения запускаются в изолированных пространствах, называемых контейнерами.

\textbf{Конфигурационный файл (Configuration File)} --- файл, используемый для настройки параметров компьютерной программы или операционной системы.

\textbf{Метаданные (Metadata)} --- данные о данных; информация, описывающая свойства других данных.

\textbf{Метамодель (Metamodel)} --- модель, описывающая структуру или правила построения других моделей.

\textbf{Облачные вычисления (Cloud Computing)} --- модель обеспечения повсеместного и удобного сетевого доступа по требованию к общему пулу конфигурируемых вычислительных ресурсов (например, сетям передачи данных, серверам, устройствам хранения данных, приложениям и сервисам), которые могут быть оперативно предоставлены и освобождены с минимальными усилиями по управлению и необходимостью взаимодействия с провайдером.

\textbf{Объектное хранилище (Object Storage)} --- архитектура хранения данных, которая управляет данными как объектами, в отличие от других архитектур хранения, таких как файловые системы, которые управляют данными как иерархией файлов, и блочные хранилища, которые управляют данными как блоками внутри секторов и дорожек.

\textbf{Оркестрация контейнеров (Container Orchestration)} --- автоматизация развертывания, масштабирования и управления контейнеризированными приложениями.

\textbf{Парсер (Parser)} --- часть компилятора или интерпретатора, отвечающая за синтаксический анализ входной последовательности символов (например, исходного кода) с целью построения структуры данных, обычно дерева разбора или абстрактного синтаксического дерева.

\textbf{Платформа данных (Data Platform)} --- интегрированный набор технологий, используемых для сбора, хранения, обработки, анализа и управления данными.

\textbf{Потоковая обработка данных (Stream Processing)} --- парадигма обработки данных, при которой данные обрабатываются непрерывно по мере их поступления, а не пакетами.

\textbf{Развертывание (Deployment)} --- процесс установки, настройки и активации программного обеспечения или инфраструктуры в целевой среде, делая его доступным для использования.

\textbf{Репозиторий (Repository)} --- место, где хранятся и поддерживаются какие-либо данные. Часто используется в контексте систем управления версиями (например, Git-репозиторий).

\textbf{Скрипт (Script)} --- программа или последовательность инструкций, которая автоматизирует выполнение задач.

\textbf{СУБД (Система Управления Базами Данных)} --- совокупность программных и языковых средств, предназначенных для создания, ведения и совместного использования баз данных многими пользователями.

\textbf{Топик (Topic)} --- в Apache Kafka, именованная категория или канал, в который продюсеры публикуют сообщения и из которого консьюмеры читают сообщения.

\textbf{Управляемый сервис (Managed Service)} --- сервис, предоставляемый облачным провайдером, который берет на себя задачи по управлению, обслуживанию и масштабированию базовой инфраструктуры этого сервиса.

\textbf{Хранилище данных (Data Warehouse)} --- предметно-ориентированная, интегрированная, привязанная ко времени и неизменяемая совокупность данных, предназначенная для поддержки принятия управленческих решений.

\textbf{CDC (Change Data Capture)} --- захват изменений данных; процесс отслеживания изменений в источнике данных (например, базе данных) и доставки этих изменений в другие системы или хранилища.

\textbf{ClickHouse} --- быстрая аналитическая СУБД с открытым исходным кодом, работающая на основе столбцового хранения данных.

\textbf{dpd (Data Platform Deployer)} --- разрабатываемый инструмент для автоматической генерации конфигураций и развертывания платформы данных.
