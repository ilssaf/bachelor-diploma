\chapter*{Постановка задачи} %
\addcontentsline{toc}{chapter}{Постановка задачи}
В современных условиях стремительного роста объёмов данных и усложнения архитектуры информационных систем ручное конфигурирование инфраструктуры для работы с большими данными становится неэффективным и подверженным ошибкам. Существующие инструменты автоматизации, такие как Terraform и Ansible, предоставляют общие механизмы развёртывания, но не предлагают специализированных решений для технологий Big Data, требующих согласованной настройки множества взаимосвязанных компонентов: систем хранения, потоковой обработки, ETL-конвейеров и инструментов визуализации.

\textbf{Целью} данной работы является разработка программного инструмента \texttt{Data Platform Deployer} (далее \texttt{dpd}), автоматизирующего процесс создания конфигурационных файлов для развёртывания платформы обработки данных на основе декларативного описания её компонентов пользователем. Инструмент принимает на вход описание целевой инфраструктуры в формате YAML и генерирует готовые к использованию артефакты:

\begin{itemize}
    \item конфигурационные файлы \texttt{docker-compose.yml} для быстрого развёртывания всех необходимых сервисов в контейнерной среде;
    \item скрипты инициализации и базовые конфигурационные файлы для СУБД (PostgreSQL\cite{postgresql}, ClickHouse\cite{clickhouse}), адаптированные под типовые задачи обработки данных;
    \item конфигурации для S3-совместимого хранилища\cite{s3};
    \item конфигурации для брокера сообщений Apache Kafka\cite{kafka}, включая создание топиков и настройки Kafka Connect с необходимыми коннекторами (Debezium для CDC, S3 Sink Connector);
    \item конфигурации для AKHQ — инструмента мониторинга Kafka и Kafka Connect\cite{kafka_connect};
    \item базовые настройки для подключения BI-систем (например, Apache Superset\cite{superset}) к развернутым источникам данных.
\end{itemize}

При разработке инструмента \texttt{dpd} должны соблюдаться следующие критерии:
\begin{itemize}
    \item Воспроизводимость: идентичные конфигурации при одинаковом входном описании.
    \item Масштабируемость: поддержка добавления новых типов компонентов через модули.
    \item Согласованность: автоматическая проверка зависимостей между сервисами.
\end{itemize}

Для достижения поставленной цели были определены следующие задачи:

\begin{enumerate}[1.]
    \item Анализ предметной области и существующих подходов к развёртыванию платформ Big Data:
          \begin{itemize}
              \item изучение типовых архитектурных паттернов платформ для обработки больших данных\cite{narkhede_kafka}
              \item исследование возможностей и ограничений существующих инструментов управления конфигурациями и IaC в контексте Big Data;
              \item определение ключевых компонентов и их типовых конфигураций для включения в состав \texttt{dpd}.
          \end{itemize}

    \item Проектирование метамодели декларативного описания инфраструктуры:
          \begin{itemize}
              \item разработка структуры YAML-файла для описания компонентов платформы, их параметров и взаимосвязей;
              \item проектирование системы валидации входных конфигураций на основе JSON Schema\cite{json_schema} для обеспечения корректности пользовательского ввода.
          \end{itemize}

    \item Разработка ядра генератора конфигураций (\texttt{dpd}):
          \begin{itemize}
              \item реализация логики парсинга входного YAML-описания;
              \item создание механизма шаблонизации для генерации конфигурационных файлов (\texttt{docker-compose.yml}, настройки сервисов и др.).
          \end{itemize}

    \item Реализация модулей генерации для ключевых компонентов платформы:
          \begin{itemize}
              \item модуль для Apache Kafka и Kafka Connect (включая коннекторы Debezium PostgreSQL, S3 Sink);
              \item модуль для СУБД PostgreSQL (источник данных);
              \item модуль для аналитической СУБД ClickHouse (хранилище данных);
              \item модуль для S3-совместимого хранилища Minio (архивное хранилище/Data Lake);
              \item модули для вспомогательных инструментов: AKHQ (мониторинг Kafka), Apache Superset (BI).
          \end{itemize}

    \item Тестирование и валидация инструмента \texttt{dpd}:
          \begin{itemize}
              \item развёртывание тестовых стендов с различной конфигурацией при помощи сгенерированных артефактов;
              \item функциональное тестирование развернутых платформ для проверки корректности работы и взаимодействия компонентов;
              \item сравнительный анализ времени и сложности развёртывания платформы с использованием \texttt{dpd} и традиционных ручных методов.
          \end{itemize}
\end{enumerate}
