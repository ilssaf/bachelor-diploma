\chapter*{Заключение} \label{ch-conclusion}
\addcontentsline{toc}{chapter}{Заключение}	% в оглавление 
В рамках настоящей дипломной работы была поставлена и успешно решена актуальная задача автоматизации процесса развертывания и конфигурирования многокомпонентных платформ для работы с большими данными. Актуальность данной задачи обусловлена возрастающей сложностью современных стеков технологий Big Data, требующих значительных временных и экспертных затрат на их первоначальную настройку и последующую поддержку. Ручной процесс конфигурирования подвержен ошибкам, затрудняет воспроизводимость и масштабирование инфраструктуры.

Основной целью работы являлась разработка программного инструмента, способного автоматизировать генерацию конфигураций и артефактов развертывания для типовых компонентов платформ данных на основе декларативного описания.

В ходе выполнения дипломной работы были достигнуты следующие ключевые результаты:

\begin{enumerate}[1.]
    \item Проведен анализ существующих подходов и инструментов для управления конфигурациями и развертывания инфраструктуры, выявлены их преимущества и недостатки применительно к задачам построения платформ больших данных. Это позволило обосновать необходимость разработки специализированного инструмента.
    \item Спроектирован и разработан программный инструмент dpd (Data Platform Deployer). Инструмент использует декларативный подход: пользователь описывает желаемую конфигурацию платформы данных (включая такие компоненты, как PostgreSQL, ClickHouse, Apache Kafka, Kafka Connect с различными коннекторами типа Debezium и S3 Sink, объектное хранилище Minio и систему визуализации Apache Superset) в едином YAML-файле. На основе этого описания dpd автоматически генерирует:
          \begin{itemize}
              \item Файл docker-compose.yml для оркестрации сервисов.
              \item Необходимые конфигурационные файлы для каждого сервиса.
              \item Скрипты инициализации баз данных и других компонентов.
              \item Обеспечивает корректную настройку сетевых взаимодействий и передачу параметров между сервисами.
          \end{itemize}
    \item Реализована поддержка расширяемости за счет модульной архитектуры и использования шаблонизатора, что позволяет в будущем добавлять поддержку новых компонентов и кастомизировать генерируемые конфигурации. (Если вы использовали ANTLR для DSL, то стоит упомянуть: "Для обработки входного описания была разработана грамматика и использован генератор парсеров ANTLR, что обеспечивает гибкость и валидацию пользовательского ввода").
    \item Проведено исследование разработанного продукта, ключевым элементом которого стала апробация инструмента dpd в промышленной среде компании ПАО «Магнит». В ходе апробации инструмент использовался для оперативного развертывания тестового стенда с целью тестирования IcebergSinkConnector для Apache Kafka. Результаты апробации, зафиксированные в Акте внедрения (опытной эксплуатации) (Приложение 3), подтвердили практическую применимость и эффективность разработанного решения.
\end{enumerate}
Все положения, выносимые на защиту, были подтверждены в ходе исследования:

\begin{itemize}
    \item Автоматическая генерация конфигураций элементов инфраструктуры программных систем для работы с большими данными: продемонстрирована способность dpd генерировать полный набор корректных артефактов на основе декларативного описания, что было подтверждено в ходе апробации.
    \item Значительное снижение трудоемкости и времени развертывания комплексной платформы данных: экспертная оценка специалистов ПАО «Магнит» показала, что использование dpd существенно сокращает время и усилия, необходимые для подготовки инфраструктуры, по сравнению с ручным подходом.
    \item Обеспечение корректности, согласованности и воспроизводимости конфигурации взаимосвязанных компонентов платформы данных: практическое применение dpd показало, что сгенерированные конфигурации обеспечивают корректное взаимодействие всех компонентов "из коробки", а сам процесс развертывания становится полностью воспроизводимым.
\end{itemize}
Практическая значимость работы заключается в создании инструмента, который упрощает и ускоряет процесс развертывания платформ больших данных, снижает порог входа для инженеров, уменьшает количество ошибок, связанных с человеческим фактором, и способствует стандартизации конфигураций. Это особенно ценно в условиях динамично развивающихся проектов и при необходимости частого создания тестовых или демонстрационных окружений.


Разработанный инструмент dpd обладает потенциалом для дальнейшего развития, включая:
\begin{itemize}
    \item Расширение списка поддерживаемых компонентов и облачных сервисов.
    \item Интеграцию с системами CI/CD для полной автоматизации жизненного цикла инфраструктуры.
    \item Разработку графического пользовательского интерфейса для упрощения описания конфигураций.
    \item Более глубокую кастомизацию генерируемых скриптов и конфигураций.
\end{itemize}

Таким образом, цели, поставленные в дипломной работе, были полностью достигнуты. Разработанный программный инструмент dpd представляет собой законченное решение, обладающее как теоретической новизной в части подхода к автоматической генерации конфигураций на основе формализованной модели, так и высокой практической ценностью, подтвержденной результатами апробации в реальных условиях.



% Заключение (2 -- 5 страниц) обязательно содержит выводы по теме работы, \textit{конкретные
% предложения и рекомендации} по исследуемым вопросам. Количество общих выводов
% должно вытекать из количества задач, сформулированных во введении выпускной
% квалификационной работы.

% Предложения и рекомендации должны быть органически увязаны с выводами
% и направлены на улучшение функционирования исследуемого объекта. При разработке
% предложений и рекомендаций обращается внимание на их обоснованность,
% реальность и практическую приемлемость.

% Заключение не должно содержать новой информации, положений, выводов и
% т. д., которые до этого не рассматривались в выпускной квалификационной работе.
% Рекомендуется писать заключение в виде тезисов.

% Последним абзацем в заключении можно выразить благодарность всем людям, которые помогали автору в написании ВКР.