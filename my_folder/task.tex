%%%% Начало оформления заголовка - оставить без изменений !!! %%%%
\input{my_folder/task_settings}	% настройки - начало 

{%\normalfont %2020
	\MakeUppercase{\SPbPU}}\\
\institute

\par}\intervalS% завершает input

\noindent
\begin{minipage}{\linewidth}
	\vspace{\mfloatsep} % интервал 	
	\begin{tabularx}{\linewidth}{Xl}
		 & УТВЕРЖДАЮ                                                                                      \\
		 & \HeadTitle                                                                                     \\
		 & \underline{\hspace*{0.1\textheight}} \Head                                                     \\
		 & <<\underline{\hspace*{0.05\textheight}}>> \underline{\hspace*{0.1\textheight}} \approveYear г. \\
	\end{tabularx}
	\vspace{\mfloatsep} % интервал 	
\end{minipage}

\intervalS{\centering\bfseries%

	ЗАДАНИЕ\\
	на выполнение %с 2020 года 
	%по выполнению % до 2020 года
	выпускной квалификационной работы


	\intervalS\normalfont%

	студенту \uline{\AuthorFullDat{} гр.~\group}


	\par}\intervalS%
%%%%
%%%% Конец оформления заголовка  %%%%



\begin{enumerate}[1.]
	\item Тема работы: {\expandafter \ulined \thesisTitle.}
	      %\item Тема работы (на английском языке): \uline{\thesisTitleEn.} % вероятно после 2021 года
	\item Срок сдачи студентом законченной работы: \uline{\thesisDeadline.}
	\item Исходные данные по работе:%
	      % \printbibliographyTask % печать списка источников % КОММЕНТИРУЕМ ЕСЛИ НЕ ИСПОЛЬЗУЕТСЯ
	      % В СЛУЧАЕ, ЕСЛИ НЕ ИСПОЛЬЗУЕТСЯ МОЖНО ТАКЖЕ ЗАЙТИ В setup.tex и закомментировать \vspace{-0.28\curtextsize}
	      \begin{itemize}
		      \item Декларативные конфигурационные файлы в формате YAML, задаваемые пользователем (инженером данных) для описания инфраструктуры обработки данных
		      \item Демонстрационный датасет, включающий тестовые данные, хранящиеся в различных источниках (PostgreSQL, S3) и обрабатываемые в системе
		      \item Автоматически сгенерированные конфигурационные файлы
	      \end{itemize}
	\item Инструментальные средства:
	      \begin{itemize}
		      \item Языки программирования: Python
		      \item Форматы конфигурационных файлов: YAML, JSON
		      \item Среда разработки: VS Code
		      \item Система управления версиями: Git
		      \item Средства контейнеризации и оркестрации: Docker, Docker Compose
		      \item Платформы потоковой обработки данных: Apache Kafka, Kafka Connect (Debezium)
		      \item Системы управления базами данных (СУБД): PostgreSQL, ClickHouse
		      \item BI-инструмент: Apache Superset
	      \end{itemize}
	\item Содержание работы (перечень подлежащих разработке вопросов):
	      \begin{enumerate}[label=\theenumi\arabic*.]
		      \item Введение.
		      \item Постановка задачи.
		      \item Обзор существующих решений.
		      \item Введение в предметную область.
		      \item Разработка инструмента
		      \item Проектирование и реализация инфраструктуры для работы с большими данными
		      \item Исследование разработанного продукта
		      \item Заключение
	      \end{enumerate}
	      Ключевые источники литературы:
	      \begin{itemize}
		      \item Альфред Ахо, Рави Сети, Джеффри Ульман. Раскрутка // Компиляторы: принципы, технологии и инструменты = Compilers: Principles, Techniques, and Tools. — М.: Вильямс, 2003. — С. 681—684. — 768 с. — ISBN 5-8459-0189-8.
		      \item Фаулер М. Непрерывная поставка: Надежная автоматизация сборки, тестирования и развертывания программного обеспечения = Continuous Delivery: Reliable Software Releases through Build, Test, and Deployment Automation. — М.: Вильямс, 2011. — 432 с. — ISBN  978-5-8459-1739-3
		      \item Таненбаум Э., ван Стин М. Распределенные системы: принципы и парадигмы = Distributed Systems: Principles and Paradigms. — 2-е изд. — М.:ДМК Пресс, 2021. — 584 с. — ISBN 978-5-97060-708-4  
	      \end{itemize}
	      % \item Перечень графического материала (с указанием обязательных чертежей):
	      %       \begin{enumerate}[label=\theenumi\arabic*.]
	      % 	      \item Схема работы метода/алгоритма.
	      % 	      \item Архитектура разработанной программы/библиотеки.
	      %       \end{enumerate}
	% \item Консультанты по работе\:
	%       \begin{enumerate}[label=\theenumi\arabic*.]
	% 	      \item  \uline{\emakefirstuc{\ConsultantExtraDegree}, \ConsultantExtra.} % закомментировать при необходимости, идёт первый по порядку.
	% 	      \item \uline{\emakefirstuc{\ConsultantNormDegree}, \ConsultantNorm{} (нормоконтроль).} %	Обязателен для всех студентов
	%       \end{enumerate}
	\item Дата выдачи задания: \uline{\thesisStartDate.}
\end{enumerate}

\intervalS%можно удалить пробел

Руководитель ВКР \uline{\hspace*{0.325\textheight} \Supervisor}


\intervalS%можно удалить пробел

Консультант  \uline{\hspace*{0.38\textheight}\ConsultantExtra}


\intervalS%можно удалить пробел

%Консультант по нормоконтролю \uline{\hspace*{0.1\textheight} \ConsultantNorm}%ПОКА НЕ ТРЕБУЕТСЯ, Т.К. ОН У ВСЕХ ПО УМОЛЧАНИЮ

Задание принял к исполнению

\intervalS%можно удалить пробел

Студент \uline{\hspace*{0.41\textheight}  \Author}



\input{my_folder/task_settings_restore}	% настройки - конец