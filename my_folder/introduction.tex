\chapter*{Введение} % * не проставляет номер
\addcontentsline{toc}{chapter}{Введение} % вносим в содержание
В современных условиях стремительного роста объемов данных и усложнения архитектуры информационных систем,
ручное конфигурирование инфраструктуры для работы с большими данными становится неэффективным и подверженным ошибкам.
Существующие инструменты автоматизации такие как Terraform, Ansible предоставляют общие механизмы развертывания,
но не предлагают специализированных решений для технологий Big Data, требующего согласованной настройки множества взаимосвязанных компонентов:
систем хранения, потоковой обработки, ETL-конвейеров и инструментов визуализации.

\textbf{Целью} данной работы является разработка программного инструмента Data Platform Deployer (далее dpd),
который автоматизирует процесс создания конфигурационных файлов для развертывания платформы обработки данных
на основе декларативного описания ее компонентов пользователем. Инструмент dpd должен принимать на вход описание целевой инфраструктуры
в формате YAML и на его основе генерировать готовые к использованию артефакты, такие как:

\begin{itemize}
	\item Конфигурационные файлы docker-compose.yml для быстрого развертывания всех необходимых сервисов в контейнерной среде.
	\item Скрипты инициализации и базовые конфигурационные файлы для СУБД (PostgreSQL, ClickHouse), адаптированные для типовых задач обработки данных.
	\item Конфигурации для S3-совместимого хранилища (например, Minio).
	\item Конфигурации для брокера сообщений Apache Kafka, включая создание топиков и настройки Kafka Connect с необходимыми коннекторами (например, Debezium для CDC, S3 Sink Connector).
	\item Конфигурации для AKHQ - инструмента для мониторинга Kafka и Kafka Connect
	\item Базовые конфигурации для подключения BI-систем (например, Apache Superset) к развернутым источникам данных.
\end{itemize}



При этом должны соблюдаться следующие критерии качества:
\begin{itemize}
	\item Воспроизводимость: идентичные конфигурации при одинаковом входном описании.
	\item Масштабируемость: поддержка добавления новых типов компонентов через модули.
	\item Согласованность: автоматическая проверка зависимостей между сервисами.
\end{itemize}

Для достижения поставленной цели и разработки инструмента dpd были определены следующие основные задачи исследования и разработки:
\begin{enumerate}[1.]
	\item Анализ предметной области и существующих подходов к развертыванию платформ данных:
	      \begin{enumerate}[a.]
		      \item Изучение типовых архитектурных паттернов платформ для обработки больших данных [8].
		      \item Исследование возможностей и ограничений существующих инструментов управления конфигурациями и IaC (Infrastructure as Code) применительно к технологиям Big Data .
		      \item Определение ключевых компонентов и их типовых конфигураций для включения в инструмент dpd.
	      \end{enumerate}

	\item Проектирование метамодели декларативного описания инфраструктуры:
	      \begin{enumerate}[a.]
		      \item Разработка структуры YAML-файла для описания компонентов платформы, их параметров и взаимосвязей.
		      \item Проектирование системы валидации входных конфигураций (например, с использованием JSON Schema [9]) для обеспечения корректности пользовательского ввода.
	      \end{enumerate}
	\item Разработка ядра генератора конфигураций инструмента dpd:
	      \begin{enumerate}[a.]
		      \item Реализация логики парсинга входного YAML-описания.
		      \item Создание механизма шаблонизации для генерации конфигурационных файлов (docker-compose.yml, настройки сервисов и т.д.).
	      \end{enumerate}
	\item Реализация модулей генерации для ключевых компонентов платформы данных:
	      \begin{enumerate}[a.]
		      \item Модуль для Apache Kafka и Kafka Connect (включая Debezium PostgreSQL, S3 Sink).
		      \item Модули для СУБД PostgreSQL (источник данных).
		      \item Модуль для аналитической СУБД ClickHouse (хранилище данных).
		      \item Модуль для S3-совместимого хранилища Minio (архивное хранилище/data lake).
		      \item Модули для вспомогательных инструментов: AKHQ для мониторинга Kafka, Apache Superset для BI
	      \end{enumerate}
	\item Тестирование и валидация разработанного инструмента dpd:
	      \begin{enumerate}[a.]
		      \item Развертывание тестовых стендов различной конфигурации с использованием сгенерированных dpd артефактов.
		      \item Проведение функционального тестирования развернутых платформ для проверки корректности их работы и взаимодействия компонентов.
		      \item Сравнительный анализ времени и сложности развертывания платформы с использованием dpd и традиционными ручными методами.
	      \end{enumerate}
\end{enumerate}


%% Вспомогательные команды - Additional commands
%\newpage % принудительное начало с новой страницы, использовать только в конце раздела
%\clearpage % осуществляется пакетом <<placeins>> в пределах секций
%\newpage\leavevmode\thispagestyle{empty}\newpage % 100 % начало новой строки