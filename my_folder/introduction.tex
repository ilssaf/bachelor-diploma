\chapter*{Введение} % * не проставляет номер
\addcontentsline{toc}{chapter}{Введение} % вносим в содержание
Современный этап развития цифровых технологий характеризуется стремительным увеличением объемов данных, генерируемых в различных сферах человеческой деятельности. Этот феномен, известный как "информационный взрыв", требует принципиально новых подходов к обработке, хранению и анализу информации. В условиях, когда традиционные методы работы с данными становятся неэффективными, особую актуальность приобретают технологии больших данных (Big Data), предлагающие инновационные решения для извлечения ценных знаний из огромных массивов неструктурированной информации[4].

\textbf{Актуальность} темы данного исследования обусловлена несколькими ключевыми факторами. Во-первых, в эпоху цифровой экономики данные становятся стратегическим ресурсом, сравнимый по значимости с традиционными материальными активами[5]. Во-вторых, сложность современных информационных систем достигла такого уровня, когда ручная настройка их компонентов становится не только трудоемкой, но и потенциально подверженной человеческим ошибкам. В-третьих, переход к agile-методологиям и DevOps-практикам требует новых подходов к управлению инфраструктурой, обеспечивающих скорость, надежность и воспроизводимость развертывания сложных систем.

В контексте этих вызовов особое значение приобретает автоматизация процессов настройки и конфигурирования инфраструктуры для работы с большими данными. Концепция "Инфраструктура как код"(Infrastructure As Code)[6] предлагает подходы для управления и предоставления вычислительной инфраструктуры с помощью декларативных или скриптовых определений.

Традиционные подходы, основанные на ручном редактировании конфигурационных файлов и скриптов, не только требуют значительных временных затрат, но и создают риски возникновения "дрейфа конфигураций" (configuration drift)[7], когда фактическое состояние системы постепенно расходится с документально зафиксированным.



%% Вспомогательные команды - Additional commands
%\newpage % принудительное начало с новой страницы, использовать только в конце раздела
%\clearpage % осуществляется пакетом <<placeins>> в пределах секций
%\newpage\leavevmode\thispagestyle{empty}\newpage % 100 % начало новой строки