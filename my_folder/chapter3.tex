\chapter{Разработка инструмента} \label{ch3}

Предварительно важно определить чёткую последовательность этапов,
которая позволит организовать работу над инструментом системно и управляемо.
Это поможет сократить риски неопределенности, оптимально распределить ресурсы и обеспечить своевременный контроль качества на каждом шаге.

Хорошим стилем является наличие введения к главе. Во введении может быть описана цель написания главы, а также приведена краткая структура главы.

\section{План разработки инструмента} \label{ch3:plan_debeloping}
\begin{enumerate}[label=\textbf{Этап \arabic*.}]
    \item Сбор и анализ требований
          На этом этапе формализуются функциональные и нефункциональные требования: определяется, какие компоненты Big Data стека поддерживаются, в каком формате задается исходный YAML, какие конфигурационные артефакты должны генерироваться.
    \item  Проектирование архитектуры
          Разрабатывается модульная архитектура инструмента, включающая парсер входного описания, генератор промежуточного представления (AST), набор шаблонов для конфигураций инструментов и механизм их объединения в итоговые файлы. Определяются границы ответственности каждого модуля, протокол взаимодействия между ними и формат плагинов для расширения функциональности.
    \item  Определение языка декларативного описания
          Уточняются синтаксис и семантика входного YAML: структура разделов, типы параметров, возможные зависимости и проверки корректности. Разрабатывается схема jsonschema для валидации пользовательских описаний на раннем этапе.
    \item  Реализация ядра: парсер и промежуточное представление
          Пишется компонент, который читает декларативный файл, проводит его валидацию по схеме (jsonschema), конструирует внутреннее дерево объектов (AST) с отображением всех сущностей и их связей. Этот модуль обеспечивает основу для дальнейших операций по генерации конфигураций.
    \item Разработка генераторов конфигурационных файлов
          На основе AST реализуются плагины-генераторы для каждого типа артефакта:
          \begin{itemize}
              \item docker-compose.yaml с сервисами и сетями;
              \item Конфигурации PostgreSQL (postgresql.conf, init.sql);
              \item JSON-файлы коннекторов Debezium и S3Sink;
              \item Файлы настроек для ClickHouse;
              \item Конфигурационные файлы для AKHQ и Superset.
          \end{itemize}
          Каждый генератор использует шаблонизатор и преобразует параметры из AST в конкретные строки и блоки файлов.
    \item Создание CLI-интерфейса
          Реализуется утилита командной строки, позволяющая пользователю запускать генерацию: передавать путь к входному файлу,  указывать директорию вывода, включать опции валидации и отладки. CLI обеспечивает удобство использования инструмента в скриптах и CI/CD-пайплайнах [28].
    \item Модуль тестирования и валидации
          Пишутся автоматические тесты: модульные тесты для парсера и генераторов, интеграционные — для проверки корректного результата генерации по ряду типовых YAML-конфигураций. Добавляются проверки на соответствие с эталонными файлами и на корректность в Docker-среде (например, пробный запуск docker-compose up).
    \item Документирование и примеры
          Готовится подробная документация: описание формата входного файла, руководство пользователя CLI, схемы и примеры конфигураций «из коробки» для типовых сценариев (EDW на PostgreSQL→Kafka→ClickHouse→Superset). Включаются рекомендации по расширению и отладке.
    \item Пилотное развертывание и сбор обратной связи
          Инструмент разворачивается в тестовой среде или локально на реальных примерах, собираются отзывы от пользователей — инженеров и аналитиков. На основе полученных замечаний корректируются шаблоны, схемы и UX CLI.
    \item Релиз и сопровождение
          Формируется релизная сборка, обеспечивается публикация в открытый репозиторий GitHub, настраивается процесс выпуска обновлений и приёма issue. Определяется модель поддержки: дорожная карта, приоритеты новых возможностей и исправлений.
\end{enumerate}


\section{Язык декларативного описания (DSL)} \label{ch3:dsl}
Практический опыт показывает, что повышение уровня абстракции и учёт специфики предметной области наиболее эффективно достигаются через разработку собственного языка предметной области - Domain Specific Language, DSL[3]. Такой язык представляет собой формальный аппарат, работающий непосредственно с понятиями и структурами предметной области, позволяя лаконично формулировать и решать большинство типовых задач.

В нашем случае DSL строится на основе YAML[30] – удобного человеко-ориентированного формата сериализации, концептуально схожего с языками разметки, но оптимизированного для записи и чтения распространённых структур данных.


Ключевые особенности синтаксиса YAML:
\begin{enumerate}[1.]
    \item Отступы и вложенность\\
          Используются только пробелы (обычно 2 или 4) для обозначения уровней вложенности. Символ табуляции запрещён.
    \item Пары «ключ–значение»\\
          Каждая запись имеет вид ключ: значение, где после двоеточия обязательно идёт пробел.
    \item Списки\\
          Элементы маркируются дефисом и пробелом (- элемент).
    \item Многострочные литералы\\
          Символ | сохраняет все разрывы строк. Символ > объединяет строки, заменяя отступы и разрывы единичными пробелами.
    \item Якоря и ссылки\\
          Якорь (\&имя) позволяет дать имя блоку значений. Ссылка (*имя) повторно вставляет ранее объявленный блок.\\
          Пример:
\end{enumerate}

\begin{verbatim}
        default: &base
        имя: Oleg
        возраст: 27

        user_2:
        <<: *base
\end{verbatim}
Чтобы формализовать синтаксис DSL и задать конечное описание потенциально
бесконечного множества допустимых конфигураций,
мы опираемся на контекстно-свободную грамматику $G=\langle N,T,R,S\rangle$, где:\\
$N$ – множество нетерминальных символов \\
$T$ – терминальные (т. е. реальные лексемы) \\
$R$ – правило вида $A$→$\alpha$ (замена нетерминала A на строку символов $\alpha$) \\
$S$ – стартовый нетерминал.

По классификации Хомского такая грамматика относится ко второму типу (КСГ): в каждом правиле слева стоит ровно один нетерминал, который может быть заменён на любую допустимую цепочку из  $A\cup B$.

Реализация парсера и генератора AST (абстрактного синтаксического дерева) опирается на ANTLR (ANother Tool for Language Recognition)[31].
Лексические правила (начинаются с большой буквы) описывают, как разбить входной текст на токены.
Пример:
\begin{verbatim}
// Лексическое правило для целых чисел
INT : [0-9]+ ; 
\end{verbatim}
Синтаксические правила (начинаются с маленькой буквы) задают структуры из токенов.
Пример:
\begin{verbatim}
// Синтаксическое правило для списка аргументов
args : expr (',' expr)* ; 

\end{verbatim}
Для группировки, повторений и альтернатив в ANTLR применяются:\\
«()» – группировка\\
«*» – 0 или более повторений\\
« +» – 1 или более\\
«?» – 0 или 1 раз\\
« |» – выбор одной из альтернатив\\
«:» и «;» – разделители начала и конца правил.\\

Описание языка DPD (Data Platform Deployer)

Язык DPD  разработан для  декларативного описания архитектуры платформы данных единым удобным форматом и автоматической генерации всех необходимых инструментов для быстрого развертывания и тестирования готового стенда. В общих чертах имеет следующую структуру:
\begin{verbatim}
project:
    name: data-platform-14
    version: 1.2.0
    description: This is a project for testing data platform
sources:
    - type: postgres
    name: postgres_1
    - type: postgres
    name: postgres_2
    - type: s3
    name: s3_1
streaming:
    kafka:
        num_brokers: 3
    connect:
        name: connect-1
storage:
    clickhouse:
    name: clickhouse-1
bi:
    superset:
    name: superset-1
\end{verbatim}
В таблице (?)  приведена часть полной грамматики, описывающая
выборочные правила.\\

\begin{longtable}{|p{3in}|p{3in}|}
    \hline
    Грамматика ANTLR4            & Комментарий                                                                                                                                                           \\ \hline
    \begin{minipage}{2.6in}
        \begin{verbatim}
grammar ConfigDSL;
        \end{verbatim}
    \end{minipage}      &                                                                                                                                                                               \\ \hline
    \begin{minipage}{3in}
        \begin{verbatim}
configFile
: projectDef
sourcesDef
streamingDef
storageDef
biDef EOF
;
\end{verbatim}
    \end{minipage}        &
    \begin{minipage}{3in}Корневое правило: определяет общую структуру конфигурационного файла, состоящего из последовательных блоков\end{minipage}                                                       \\ \hline
    \begin{minipage}{2.6in}
        \begin{verbatim}
projectDef
: PROJECT COLON NAME COLON 
STRING VERSION COLON STRING 
DESCRIPTION COLON STRING
;
    \end{verbatim}
    \end{minipage} &
    \begin{minipage}{3in}Правило для секции "project": описывает метаданные проекта\end{minipage}                                                                                                        \\ \hline
    \begin{minipage}{3in}
        \begin{verbatim}
sourcesDef
: SOURCES COLON 
sourceItem+
        ;
    \end{verbatim}
    \end{minipage}
                                 &
    \begin{minipage}{2.6in}
        Правило для секции «sources»: определяет список источников данных.
    \end{minipage}
    \\ \hline

    \begin{minipage}{3in}
        \begin{verbatim}
sourceItem
: DASH NAME COLON STRING 
TYPE COLON sourceType
( PORT COLON NUMBER )?
( USERNAME COLON STRING )?
( PASSWORD COLON STRING )?
( ACCESS_KEY COLON STRING )?
( SECRET_KEY COLON STRING )?
( REGION COLON STRING )?
( BUCKET COLON STRING )?
;
    \end{verbatim}
    \end{minipage}
                                 &
    \begin{minipage}{2.6in}
        Правило для описания одного источника данных: имя, тип (Postgres/S3) и опциональные параметры (порт, учётные данные, детали S3).
    \end{minipage}
    \\ \hline

    \begin{minipage}{3in}
        \begin{verbatim}
sourceType
: POSTGRES
| S3
;
    \end{verbatim}
    \end{minipage}
                                 &
    \begin{minipage}{2.6in}
        Правило для определения типа источника данных (PostgreSQL или S3).
    \end{minipage}
    \\ \hline

    \begin{minipage}{3in}
        \begin{verbatim}
streamingDef
: STREAMING COLON 
(kafkaDef | connectDef)+
        ;
    \end{verbatim}
    \end{minipage}
                                 &
    \begin{minipage}{2.6in}
        Правило для секции «streaming»: задаёт компоненты потоковой обработки (Kafka или Kafka Connect).
    \end{minipage}
    \\ \hline

    \begin{minipage}{3in}
        \begin{verbatim}
kafkaDef
: KAFKA COLON 
NUM_BROKERS COLON 
NUMBER
;
    \end{verbatim}
    \end{minipage}
                                 &
    \begin{minipage}{2.6in}
        Правило для конфигурации Kafka: число брокеров.
    \end{minipage}
    \\ \hline

    \begin{minipage}{3in}
        \begin{verbatim}
connectDef
: CONNECT COLON 
NAME COLON STRING
        ;
    \end{verbatim}
    \end{minipage}
                                 &
    \begin{minipage}{2.6in}
        Правило для конфигурации Kafka Connect: имя инстанса.
    \end{minipage}
    \\ \hline

    \begin{minipage}{3in}
        \begin{verbatim}
storageDef
: STORAGE COLON 
clickhouseDef
;

clickhouseDef
: CLICKHOUSE COLON 
NAME COLON STRING
;
    \end{verbatim}
    \end{minipage}
                                 &
    \begin{minipage}{2.6in}
        Правило для секции «storage»: задаёт компонент хранения данных и его параметры (ClickHouse).
    \end{minipage}
    \\ \hline

    \begin{minipage}{3in}
        \begin{verbatim}
biDef
: BI COLON supersetDef
;

supersetDef
: SUPERSET COLON NAME COLON STRING
    ( USERNAME COLON STRING )?
    ( PASSWORD COLON STRING )?
;
    \end{verbatim}
    \end{minipage}
                                 &
    \begin{minipage}{2.6in}
        Правило для секции «bi»: задаёт инструмент BI (Apache Superset) и его опциональные параметры.
    \end{minipage}
    \\ \hline

    \begin{minipage}{3in}
        \begin{verbatim}
// Лексемы (tokens)
PROJECT     : 'project';
SOURCES     : 'sources';
STREAMING   : 'streaming';
STORAGE     : 'storage';
BI          : 'bi';
KAFKA       : 'kafka';
CONNECT     : 'connect';
CLICKHOUSE  : 'clickhouse';
SUPERSET    : 'superset';
NAME        : 'name';
VERSION     : 'version';
DESCRIPTION : 'description';
TYPE        : 'type';
PORT        : 'port';
USERNAME    : 'username';
PASSWORD    : 'password';
ACCESS_KEY  : 'access_key';
SECRET_KEY  : 'secret_key';
REGION      : 'region';
BUCKET      : 'bucket';
NUM_BROKERS : 'num_brokers';
POSTGRES    : 'postgres';
S3          : 's3';
COLON       : ':';
DASH        : '-';
STRING      : '"' ( ~[\\"] | '\\' . )*? '"' ;
NUMBER      : [0-9]+ ;
WS          : [ \t\r\n]+ -> skip ;
    \end{verbatim}
    \end{minipage}
                                 &
    \begin{minipage}{2.6in}
        Лексемы (tokens): ключевые слова, разделители, строковые и числовые литералы, пробельные символы.                                                                                     \end{minipage} \\ \hline
\end{longtable}

%\FloatBarrier % заставить рисунки и другие подвижные (float) элементы остановиться


\section{Выводы} \label{ch3:conclusion}

Текст выводов по главе \thechapter.


%% Вспомогательные команды - Additional commands
%
%\newpage % принудительное начало с новой страницы, использовать только в конце раздела
%\clearpage % осуществляется пакетом <<placeins>> в пределах секций
%\newpage\leavevmode\thispagestyle{empty}\newpage % 100 % начало новой страницы