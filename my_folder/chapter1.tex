\chapter{Обзор существующих решений} \label{ch1}
Развертывание и управление инфраструктурой для систем обработки больших данных (Big Data) представляет собой сложную задачу, требующую координации множества разнородных компонентов, настройки их взаимодействия и обеспечения масштабируемости, надёжности и безопасности.

Современные подходы к управлению инфраструктурой стремятся автоматизировать эти процессы, однако специфика Big Data-систем накладывает дополнительные требования. В данном разделе рассматриваются существующие инструменты управления инфраструктурой и анализируются ключевые требования к интеграции компонентов, характерные для платформ обработки больших данных, с акцентом на решения, актуальные для российского рынка.

\begin{enumerate}
	\item \textbf{Управляемые облачные сервисы (Managed Cloud Services)} на примере российских провайдеров
	      Российские облачные провайдеры, такие как Yandex Cloud[номер]???вставитьлит{} и VK Cloud[номер]???вставитьлит{} (ранее Mail.ru Cloud Solutions), активно развивают свои портфели управляемых сервисов, предназначенных для работы с большими данными. Эти сервисы позволяют значительно упростить создание и обслуживание сложной инфраструктуры.

	      Предлагаемые сервисы:
	      \begin{itemize}
		      \item \textbf{Yandex Cloud:} Yandex Data Proc (управляемый сервис для Apache Spark™ и Apache Hadoop®), Yandex Managed Service for Apache Kafka®, Yandex Managed Service for ClickHouse®, Yandex Managed Service for Greenplum®, Yandex Managed Service for PostgreSQL, объектное хранилище Yandex Object Storage (совместимое с S3 API).
		      \item \textbf{VK Cloud:} управляемые базы данных (PostgreSQL, ClickHouse, MySQL и др.), сервис «Большие данные» на базе Arenadata Hadoop (ADH) и Arenadata Kafka (ADK) для пакетной и потоковой обработки, объектное хранилище (совместимое с S3 API).
	      \end{itemize}

	      Механизм генерации конфигураций: при создании и настройке управляемых сервисов через веб-консоль, CLI или API провайдера пользователь указывает высокоуровневые параметры (тип и количество вычислительных узлов, версии ПО, базовые настройки сети и параметры безопасности). Облачная платформа автоматически генерирует и поддерживает низкоуровневые конфигурации виртуальной инфраструктуры и сервисов (например, *-site.xml для Hadoop, server.properties для Kafka). Возможности кастомизации расширены через дополнительные опции или параметры запуска.

	      Преимущества:
	      \begin{itemize}
		      \item значительное упрощение развёртывания и управления: время запуска инфраструктуры сокращается с недель или дней до часов или минут;
		      \item снижение операционной нагрузки: провайдер обеспечивает обновление ПО, патчинг, мониторинг и доступность;
		      \item встроенные механизмы масштабирования и отказоустойчивости;
		      \item интеграция с сервисами экосистемы провайдера.
	      \end{itemize}

	      Недостатки:
	      \begin{itemize}
		      \item привязка к конкретному провайдеру (vendor lock-in);
		      \item ограниченная гибкость в глубоких настройках;
		      \item высокая стоимость при постоянной нагрузке;
		      \item непрозрачность детальных конфигураций.
	      \end{itemize}
	\item \textbf{Интегрированные платформы данных (Integrated Data Platforms)} на примере российских разработок
	      На российском рынке представлены комплексные решения для жизненного цикла работы с данными, от сбора и хранения до обработки и анализа. Примеры: Arenadata и CedrusData.
	      \begin{itemize}

		      \item \textbf{Arenadata:} продукты на базе открытого кода, включая Arenadata Hadoop (ADH), Arenadata DB (Greenplum), Arenadata Streaming (Kafka, NiFi), Arenadata QuickMarts (ClickHouse). Для управления развёртыванием используется Arenadata Platform Manager (ADPM).
		            Механизм генерации (ADPM): пользователь задаёт через интерфейс или декларативный файл состав кластера и ключевые параметры; ADPM генерирует и применяет конфигурации для всех компонентов, обеспечивая их согласованность.

		      \item \textbf{CedrusData:} платформа для высокопроизводительной аналитики и обработки данных, включающая распределённое хранилище, SQL-обработку и инструменты управления.
		            Механизм генерации: инсталлятор или скрипты запрашивают у администратора параметры системы, после чего генерируются и применяются конфигурации для всех компонентов платформы.
	      \end{itemize}
	      Преимущества:
	      \begin{itemize}
		      \item единая точка входа и управления;
		      \item проверенные интеграции и оптимальная совместимость;
		      \item упрощённое развёртывание и обновление;
		      \item техническая поддержка от вендора.
	      \end{itemize}

	      Недостатки:
	      \begin{itemize}
		      \item высокая стоимость лицензий и поддержки;
		      \item «чёрный ящик» внутренних настроек;
		      \item ограниченный выбор компонентов и версий;
		      \item зависимость от экосистемы вендора;
		      \item сложность освоения комплексных платформ.
	      \end{itemize}
	\item \textbf{Kubernetes Operators}
	      Операторы в Kubernetes позволяют управлять stateful-приложениями Big Data: Strimzi для Kafka, CrunchyData и Zalando для PostgreSQL, операторы для ClickHouse, Flink, Spark и др.
	      Механизм генерации: пользователь задаёт CRD (Custom Resource Definition) с высокоуровневым описанием (например, \texttt{kind: Kafka, spec:{replicas:3, storage:...}}), оператор создаёт и управляет Kubernetes-объектами (Deployments, StatefulSets, ConfigMaps, Secrets) в соответствии с этим описанием.

	      Преимущества: декларативный подход, автоматизация жизненного цикла, нативная модель управления в Kubernetes.
	      Недостатки: требует инфраструктуры и экспертизы в Kubernetes, каждый оператор соответствует одному сервису, интеграции между операторами частично ручные, не генерирует docker-compose.yml и не подходит для сред вне Kubernetes.
	\item \textbf{Шаблонизаторы и модули для инструментов управления конфигурациями}
	      Ansible, Chef, Puppet, SaltStack с шаблонами (Jinja2, ERB) и готовыми ролями/рецептами для конкретных приложений (PostgreSQL, Kafka и т.д.).
	      Механизм генерации: переменные (например, число брокеров, параметры памяти) задаются в YAML-файлах, шаблоны конфигураций используют эти переменные для генерации файлов, которые затем распространяются на узлы.

	      Преимущества: гибкость, переиспользование кода, интеграция с существующими CI/CD.
	      Недостатки: требует знания инструментов, описание системы разбросано между плейбуками, шаблонами и переменными, сложнее задать стек целиком в одном декларативном файле.
\end{enumerate}

Существующие решения предлагают разные уровни абстракции и автоматизации для развёртывания Big Data-систем.
Управляемые облачные сервисы и интегрированные платформы обеспечивают высокий уровень автоматизации ценой гибкости и привязки к поставщику. Kubernetes Operators дают декларативность, но требуют экосистемы Kubernetes и экспертизы. 
Шаблонизаторы в конфигурационных инструментах позволяют генерировать файлы, но требуют значительной ручной настройки. 

Подход, разрабатываемый в рамках данной работы, предлагает специализированный инструмент для автоматической генерации конфигураций распространённых Big Data-технологий из единого высокоуровневого YAML-файла, что снижает порог входа и ускоряет итерации при построении и тестировании платформ.

% не рекомендуется использовать отдельную section <<введение>> после лета 2020 года
%\section{Введение. Сложносоставное название первого параграфа первой главы для~демонстрации переноса слов в содержании} \label{ch1:intro}

% Хорошим стилем является наличие введения к главе, которое \textit{начинается непосредственно после названия главы, без оформления в виде отдельного параграфа}. Во введении может быть описана цель написания главы, а также приведена краткая структура главы. Например, в параграфе \ref{ch1:sec1} приведены примеры оформления одиночных формул, рисунков и таблицы. Параграф \ref{ch1:sec2} посвящён многострочным формулам и сложносоставным рисункам.

% Текст данной главы призван привести \textit{краткие} примеры оформления текстово-графических объектов. Более подробные примеры можно посмотреть в следующей главе, а также в рекомендациях студентам \cite{spbpu-student-thesis-template-author-guide}. 


% \section{Название параграфа} \label{ch1:sec1}


% \subsection{Название первого подпараграфа первого параграфа первой главы для~демонстрации переноса слов в содержании} % ~ нужен, чтобы избавиться от висячего предлога (союза) в конце строки

% Содержание первого подпараграфа первого параграфа первой главы.



% Одиночные формулы оформляют в окружении \texttt{equation}, например, как указано в следующей одиночной нумерованной формуле:
% %
% %
% \begin{equation}% лучше не оставлять пропущенную строку (\par) перед окружениями для избежания лишних отсупов в pdf
% \label{eq:Pi-ch1} % eq - equations, далее название, ch поставлено для избежания дублирования
% \pi \approx 3,141.
% \end{equation}
% %
% %
% \begin{figure}[ht!] 
% 	\center
% 	\includegraphics [scale=0.27] {my_folder/images//spbpu_hydrotower}
% 	\caption{Вид на гидробашню СПбПУ \cite{spbpu-gallery}} 
% 	\label{fig:spbpu_hydrotower}  
% \end{figure}
% %
% %
% %\begin{table} [htbp]% Пример оформления таблицы
% %	\centering\small
% %	\caption{Представление данных для сквозного примера по ВКР \cite{Peskov2004}}%
% %	\label{tab:ToyCompare}		
% %		\begin{tabular}{|l|l|l|l|l|l|}
% %			\hline
% %			$G$&$m_1$&$m_2$&$m_3$&$m_4$&$K$\\
% %			\hline
% %			$g_1$&0&1&1&0&1\\ \hline
% %			$g_2$&1&2&0&1&1\\ \hline
% %			$g_3$&0&1&0&1&1\\ \hline
% %			$g_4$&1&2&1&0&2\\ \hline
% %			$g_5$&1&1&0&1&2\\ \hline
% %			$g_6$&1&1&1&2&2\\ \hline		
% %		\end{tabular}	
% %	\normalsize% возвращаем шрифт к нормальному
% %\end{table}


% % \firef{} от figure reference
% % \taref{} от table reference
% % \eqref{} от equation reference

% На \firef{fig:spbpu_hydrotower} изображена гидробашня СПбПУ, а в \taref{tab:ToyCompare} приведены данные, на примере которых коротко и наглядно будет изложена суть ВКР.


% \section{Название параграфа} \label{ch1:sec2} 



% Формулы могут быть размещены в несколько строк. Чтобы выставить номер формулы напротив средней строки, используйте окружение \verb|multlined| из пакета \verb|mathtools| следующим образом \cite{Ganter1999}:
% %
% \begin{equation} 
% \label{eq:fConcept-order-ch1}
% \begin{multlined}
% (A_1,B_1)\leq (A_2,B_2)\; \Leftrightarrow \\  \Leftrightarrow\; A_1\subseteq A_2\; \Leftrightarrow \\ \Leftrightarrow\; B_2\subseteq B_1. 
% \end{multlined}
% \end{equation}


% Используя команду \verb|\labelcref| из пакета \verb|cleveref|, допустимо следующим образом оформлять ссылку на несколько формул:
% (\labelcref{eq:Pi-ch1,eq:fConcept-order-ch1}).
% %
% %
% \input{my_folder/tex/fig-spbpu-whitehall-three-in-one} % пример подключения 3х иллюстрации в одном рисунке

% Пример ссылок \cite{Article,Book,Booklet,Conference,Inbook,Incollection,Manual,Mastersthesis,Misc,Phdthesis,Proceedings,Techreport,Unpublished,badiou:briefings}, а также ссылок с указанием страниц, на котором отображены номера страниц  \cite[с.~96]{Naidenova2017} или в виде мультицитаты на несколько источников \cites[с.~96]{Naidenova2017}[с.~46]{Ganter1999}. Часть библиографических записей носит иллюстративный характер и не имеет отношения к реальной литературе. 



% %\FloatBarrier % заставить рисунки и другие подвижные (float) элементы остановиться

% \section{Выводы} \label{ch1:conclusion}

% Текст выводов по главе \thechapter.

% Кроме названия параграфа <<выводы>> можно использовать (единообразно по всем главам) следующие подходы к именованию последних разделов с результатами по главам:
% \begin{itemize}
% 	\item <<выводы по главе N>>, где N --- номер соответствующей главы;
% 	\item <<резюме>>;
% 	\item <<резюме по главе N>>, где N --- номер соответствующей главы.
% \end{itemize}

% Параграф с изложением выводов по главе \textit{является обязательным}.

%% Вспомогательные команды - Additional commands
%
%\newpage % принудительное начало с новой страницы, использовать только в конце раздела
%\clearpage % осуществляется пакетом <<placeins>> в пределах секций
%\newpage\leavevmode\thispagestyle{empty}\newpage % 100 % начало новой страницы