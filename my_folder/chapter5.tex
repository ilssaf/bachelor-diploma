\chapter{Исследование разработанного продукта} \label{ch5}

Целью данного раздела является исследование практической применимости и эффективности разработанного программного инструмента dpd (Data Platform Deployer) для автоматической генерации конфигураций элементов инфраструктуры программных систем для работы с большими данными. Учитывая, что dpd представляет собой инструмент генерации конфигураций и автоматизации процесса развертывания, а не систему, подвергаемую нагрузкам в runtime, традиционные методы количественного нагрузочного тестирования самого инструмента не являются релевантными для оценки его ключевых преимуществ.

Вместо этого, для оценки эффективности и подтверждения выдвигаемых на защиту положений была выбрана методология, основанная на экспертной оценке и апробации инструмента в реальных условиях. Ключевым элементом исследования стала практическая апробация dpd в производственной среде компании ПАО «Магнит», где инструмент использовался для решения конкретных инженерных задач. Результаты данной апробации и экспертные заключения специалистов компании служат основной базой для подтверждения положений, выносимых на защиту. Официальным подтверждением успешной апробации является Акт внедрения (опытной эксплуатации), представленный в Приложении \ref{myappendices::vkr_magnit}.

Исследование направлено на подтверждение следующих положений:

\begin{itemize}
    \item Эффективность автоматической генерации конфигураций элементов инфраструктуры программных систем для работы с большими данными с помощью разработанного инструмента dpd.
    \item Значительное снижение трудоемкости и времени развертывания комплексной платформы данных за счет использования разработанного декларативного подхода и инструмента автоматизации dpd.
    \item  Обеспечение корректности, согласованности и воспроизводимости конфигурации взаимосвязанных компонентов платформы данных посредством генерации единых и проверенных артефактов развертывания с помощью dpd.
\end{itemize}
\section{Подтверждение эффективности автоматической генерации конфигураций}

Первое положение, выносимое на защиту, касается самой основной функции разработанного инструмента dpd – автоматической генерации конфигураций. Инструмент спроектирован таким образом, чтобы на основе единого декларативного файла в формате YAML, описывающего желаемый состав и параметры платформы данных, генерировать весь необходимый набор артефактов:
\begin{itemize}
    \item Файл docker-compose.yml для оркестрации сервисов в контейнерах Docker.
    \item Скрипты инициализации для баз данных (например, создание пользователей, баз данных, таблиц в PostgreSQL и ClickHouse).
    \item Конфигурационные файлы для отдельных компонентов, такие как настройки брокера Kafka, параметры коннекторов Kafka Connect (включая Debezium, S3 Sink, и, как было апробировано, Iceberg Sink), конфигурации для Minio и подключения к источникам данных в Apache Superset.
\end{itemize}
В ходе апробации в ПАО «Магнит», инструмент dpd был использован для развертывания тестового стенда, включающего Apache Kafka, Minio и специфический набор коннекторов Kafka Connect, в частности, для тестирования IcebergSinkConnector. Эксперты компании подтвердили, что dpd успешно справился с задачей генерации всех необходимых конфигурационных файлов. Было отмечено, что сгенерированные конфигурации были полными, синтаксически корректными и соответствовали параметрам, указанным во входном YAML-файле. Это позволило инженерной команде быстро получить работоспособный стенд без необходимости ручного создания и отладки многочисленных конфигурационных файлов. Данный факт зафиксирован в Акте внедрения (Приложение \ref{myappendices::vkr_magnit}).

Таким образом, практическое применение dpd в условиях реальной задачи подтвердило его способность эффективно автоматизировать процесс генерации конфигураций для сложной инфраструктуры больших данных.

\section{Оценка снижения трудоемкости и времени развертывания}
Второе положение утверждает, что использование dpd приводит к значительному снижению трудоемкости и времени, необходимых для развертывания платформы данных. Эта оценка проводилась на основе экспертного мнения специалистов ПАО «Магнит», участвовавших в апробации, путем сравнения предполагаемых трудозатрат при ручной настройке аналогичного стенда и фактических затрат при использовании dpd.
Ручное развертывание многокомпонентной платформы данных, включающей PostgreSQL, ClickHouse, Kafka, Kafka Connect с различными коннекторами, Minio и Superset, является сложной и трудоемкой задачей. Она требует:
\begin{itemize}
\item Глубоких знаний конфигурационных особенностей каждого компонента.
\item Ручного создания и редактирования десятков конфигурационных параметров в различных файлах.
\item Тщательной настройки сетевых взаимодействий между сервисами.
\item Отладки возможных конфликтов и ошибок, возникающих из-за опечаток или неверных настроек.
\end{itemize}
По оценкам экспертов ПАО «Магнит», ручная настройка стенда, аналогичного тому, что был развернут для тестирования IcebergSinkConnector, могла бы занять от нескольких часов до целого рабочего дня квалифицированного инженера, особенно если учесть необходимость отладки и проверки корректности связей.

При использовании инструмента dpd процесс сводился к следующим шагам:
\begin{enumerate}[1.]
\item Описание требуемой конфигурации стенда в YAML-файле (занимает, в зависимости от сложности, от нескольких минут до получаса для опытного пользователя, знакомого с форматом).
\item Запуск dpd для генерации артефактов (занимает несколько секунд).
\item Запуск развертывания с помощью сгенерированного docker-compose.yml (занимает несколько минут на скачивание образов и запуск контейнеров).
\end{enumerate}
Таким образом, общее время на получение работоспособного стенда с помощью dpd сократилось до десятков минут. Эксперты ПАО «Магнит» отметили, что основное преимущество dpd заключается не только в прямом сокращении времени, но и в значительном снижении когнитивной нагрузки на инженера, которому больше не нужно держать в голове все детали конфигурации каждого сервиса. Это позволяет высвободить ресурсы для решения более высокоуровневых задач, таких как непосредственное тестирование функциональности (в данном случае IcebergSinkConnector). Снижение трудоемкости и экономия времени отражены в положительных отзывах, указанных в Акте внедрения (Приложение \ref{myappendices::vkr_magnit}).

\section{Обеспечение корректности, согласованности и воспроизводимости конфигураций}

Третье положение касается качества генерируемых конфигураций, а именно их корректности, согласованности и воспроизводимости.
\begin{itemize}  
\item Корректность и согласованность: Инструмент dpd реализует внутреннюю логику, которая обеспечивает согласованность настроек между различными компонентами. Например, адреса и порты сервисов, учетные данные, имена топиков Kafka, пути к хранилищам генерируются таким образом, чтобы все компоненты могли корректно взаимодействовать друг с другом без дополнительной ручной правки. В ходе апробации в ПАО «Магнит» было подтверждено, что развернутый с помощью dpd стенд функционировал корректно "из коробки": IcebergSinkConnector успешно подключался к Kafka и Minio, используя сгенерированные конфигурации, и корректно обрабатывал данные. Это свидетельствует о том, что dpd генерирует не просто набор файлов, а целостную, работоспособную систему с правильно настроенными взаимосвязями.
\item Воспроизводимость: Декларативный подход, реализованный в dpd (описание всей инфраструктуры в одном YAML-файле), и автоматизация генерации конфигураций обеспечивают высокую степень воспроизводимости. При наличии одного и того же входного YAML-файла, dpd всегда будет генерировать идентичный набор конфигурационных артефактов. Это критически важно для создания идентичных окружений (например, для разработки, тестирования, демонстраций) и для исключения проблемы "работает на моей машине". Эксперты ПАО «Магнит» отметили, что возможность быстро и гарантированно воспроизводить тестовый стенд является значительным преимуществом, особенно при итеративном тестировании и отладке новых компонентов, таких как IcebergSinkConnector. Возможность многократного развертывания идентичной конфигурации была практически подтверждена в ходе апробации.
\end{itemize}
Уменьшение вероятности человеческой ошибки – еще один важный аспект, способствующий корректности. Автоматическая генерация на основе проверенной логики и шаблонов исключает опечатки, пропущенные параметры или неверные значения, которые часто возникают при ручной настройке. Эти аспекты также были положительно оценены специалистами ПАО «Магнит» (см. Приложение \ref{myappendices::vkr_magnit}).

\section{Результаты апробации в ПАО «Магнит» и экспертная оценка}

Как уже упоминалось, ключевым этапом исследования стала апробация инструмента dpd в условиях реальной производственной задачи в компании ПАО «Магнит». Задачей апробации являлось тестирование нового коннектора IcebergSinkConnector для Apache Kafka, предназначенного для сохранения данных из Kafka в таблицы Apache Iceberg, хранящиеся в S3-совместимом хранилище (Minio).

Для решения этой задачи с помощью dpd был оперативно развернут тестовый стенд, включающий следующие компоненты, сконфигурированные для совместной работы:
\begin{itemize}  
\item Apache Kafka (брокер сообщений).
\item  Minio (S3-совместимое объектное хранилище для таблиц Iceberg).
\item  Kafka Connect с развернутым IcebergSinkConnector.
\item  PostgreSQL как источник данных для Debezium, если бы тестировался полный пайплайн CDC -> Kafka -> Iceberg.
\end{itemize}

Использование dpd позволило:
\begin{enumerate}[1.]
\item Быстро создать необходимую инфраструктурную обвязку: Инженеры смогли сосредоточиться непосредственно на логике работы и настройках самого IcebergSinkConnector, а не на развертывании и конфигурировании базовых сервисов.
\item Легко модифицировать конфигурацию стенда: При необходимости изменения параметров Kafka, Minio или других компонентов для различных сценариев тестирования, достаточно было внести правки в YAML-файл и перегенерировать конфигурацию, что занимало минимум времени.
\item Гарантировать корректность и изолированность тестового окружения: Это позволило получить надежные результаты тестирования самого коннектора.
\end{enumerate}

По итогам апробации был составлен Акт внедрения (опытной эксплуатации) (Приложение \ref{myappendices::vkr_magnit}), в котором специалисты ПАО «Магнит» подтвердили:
\begin{itemize} 
\item Успешное применение инструмента dpd для развертывания тестового стенда.
\item Значительное сокращение времени и трудозатрат на подготовку инфраструктуры.
\item Высокое удобство использования декларативного подхода для описания конфигурации.
\item Корректность и согласованность генерируемых конфигураций.
\item Практическую пользу инструмента для решения инженерных задач, связанных с тестированием и развертыванием компонентов экосистемы больших данных.
\end{itemize}
Экспертная оценка со стороны специалистов ПАО «Магнит» однозначно подтвердила состоятельность и полезность разработанного инструмента dpd.