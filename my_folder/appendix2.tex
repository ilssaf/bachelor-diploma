\chapter{SQL код для забора данных из Kafka в ClickHouse}\label{ex_1_sql}							% 

В приложениии приведен SQL код для забора данных из Kafka топика в ClickHouse для таблицы \texttt{orders}

\begin{lstlisting}[%
	language=SQL,%      язык
	numbers=left,%      нумерация строк слева
	stepnumber=1,%      номер каждой строки
	firstnumber=1,%     начинают с 1
	numberstyle=\tiny\color{gray},% стиль номеров
	numbersep=5pt,%     отступ номеров от кода
	basicstyle=\ttfamily\small,% шрифт кода
	keywordstyle=\color{blue}\bfseries,% стиль ключевых слов
	stringstyle=\color{red},% стиль строковых литералов
	commentstyle=\color{green!50!black},% стиль комментариев
	frame=single,%      рамка вокруг кода
	breaklines=true,%   автоматический перенос строк
	caption={SQL код для забора данных из Kafka в ClickHouse},%
	label={lst:clickhouse_schema}% метка для ссылки
  ]
  CREATE TABLE clickhouse_1_db.kafka_orders (data String)
  ENGINE = Kafka()
  SETTINGS kafka_broker_list = 'kafka-0:9092,kafka-1:9092,kafka-2:9092,kafka-3:9092,kafka-4:9092,kafka-5:9092',
		   kafka_topic_list  = 'postgres_1.public.orders',
		   kafka_group_name  = 'kafka-ch',
		   kafka_format      = 'JSONAsString';
  
  CREATE TABLE clickhouse_1_db.orders
  (
	 order_id            Int16,
	 customer_id         Nullable(String),
	 employee_id         Nullable(Int16),
	 order_date          Nullable(Date),
	 required_date       Nullable(Date),
	 shipped_date        Nullable(Date),
	 ship_via            Nullable(Int16),
	 freight             Nullable(Float32),
	 ship_name           Nullable(String),
	 ship_address        Nullable(String),
	 ship_city           Nullable(String),
	 ship_region         Nullable(String),
	 ship_postal_code    Nullable(String),
	 ship_country        Nullable(String)
  )
  ENGINE = MergeTree()
  ORDER BY order_id;
  
  CREATE MATERIALIZED VIEW clickhouse_1_db.kafka_orders_mv
  TO clickhouse_1_db.orders AS
  SELECT
	 CAST(JSON_VALUE(data, '$.after.order_id')              AS Int16)           AS order_id,
	 CAST(JSON_VALUE(data, '$.after.customer_id')           AS Nullable(String)) AS customer_id,
	 CAST(JSON_VALUE(data, '$.after.employee_id')           AS Nullable(Int16))  AS employee_id,
	 CAST(toDate(CAST(JSON_VALUE(data, '$.after.order_date')    AS Nullable(Int64))) AS Nullable(Date))   AS order_date,
	 CAST(toDate(CAST(JSON_VALUE(data, '$.after.required_date') AS Nullable(Int64))) AS Nullable(Date))   AS required_date,
	 CAST(toDate(CAST(JSON_VALUE(data, '$.after.shipped_date')  AS Nullable(Int64))) AS Nullable(Date))   AS shipped_date,
	 CAST(JSON_VALUE(data, '$.after.ship_via')              AS Nullable(Int16))  AS ship_via,
	 CAST(JSON_VALUE(data, '$.after.freight')               AS Nullable(Float32))AS freight,
	 CAST(JSON_VALUE(data, '$.after.ship_name')             AS Nullable(String)) AS ship_name,
	 CAST(JSON_VALUE(data, '$.after.ship_address')          AS Nullable(String)) AS ship_address,
	 CAST(JSON_VALUE(data, '$.after.ship_city')             AS Nullable(String)) AS ship_city,
	 CAST(JSON_VALUE(data, '$.after.ship_region')           AS Nullable(String)) AS ship_region,
	 CAST(JSON_VALUE(data, '$.after.ship_postal_code')      AS Nullable(String)) AS ship_postal_code,
	 CAST(JSON_VALUE(data, '$.after.ship_country')          AS Nullable(String)) AS ship_country
  FROM clickhouse_1_db.kafka_orders;
  \end{lstlisting}

% В приложении приведены формулы \eqref{eq:Pi-app}, \eqref{eq:Pi-app-}, \firef{fig:spbpu_hydrotower-app}, \firef{fig:spbpu_hydrotower-app-}, \taref{tab:ToyCompare-app}, \taref{tab:ToyCompare-app-}\footnote{Внимание! Пример оформления подстрочной ссылки (сноски).}.


% \begin{equation}% лучше не оставлять пропущенную строку (\par) перед окружениями для избежания лишних отсупов в pdf
% \label{eq:Pi-app-} % eq - equations, далее название, ch поставлено для избежания дублирования
% \pi \approx 3,141.
% \end{equation}
% %
% %
% \begin{figure}[ht!] 
% 	\center
% 	\includegraphics [scale=0.27] {my_folder/images//spbpu_hydrotower}
% 	\caption{Вид на гидробашню СПбПУ \cite{spbpu-gallery}} 
% 	\label{fig:spbpu_hydrotower-app-}  
% \end{figure}

% \begin{table} [htbp]% Пример оформления таблицы
% 	\centering\small
% 	\caption{Представление данных для сквозного примера по ВКР \cite{Peskov2004}}%
% 	\label{tab:ToyCompare-app-}		
% 	\begin{tabular}{|l|l|l|l|l|l|}
% 		\hline
% 		$G$&$m_1$&$m_2$&$m_3$&$m_4$&$K$\\
% 		\hline
% 		$g_1$&0&1&1&0&1\\ \hline
% 		$g_2$&1&2&0&1&1\\ \hline
% 		$g_3$&0&1&0&1&1\\ \hline
% 		$g_4$&1&2&1&0&2\\ \hline
% 		$g_5$&1&1&0&1&2\\ \hline
% 		$g_6$&1&1&1&2&2\\ \hline		
% 	\end{tabular}	
% 	\normalsize% возвращаем шрифт к нормальному
% \end{table}




% \section{Подраздел приложения}\label{app-2-1}							


% \begin{equation}% лучше не оставлять пропущенную строку (\par) перед окружениями для избежания лишних отсупов в pdf
% \label{eq:Pi-app} % eq - equations, далее название, ch поставлено для избежания дублирования
% \pi \approx 3,141.
% \end{equation}
% %
% %
% \begin{figure}[ht!] 
% 	\center
% 	\includegraphics [scale=0.27] {my_folder/images//spbpu_hydrotower}
% 	\caption{Вид на гидробашню СПбПУ \cite{spbpu-gallery}} 
% 	\label{fig:spbpu_hydrotower-app}  
% \end{figure}

% \begin{table} [htbp]% Пример оформления таблицы
% 	\centering\small
% 	\caption{Представление данных для сквозного примера по ВКР \cite{Peskov2004}}%
% 	\label{tab:ToyCompare-app}		
% 	\begin{tabular}{|l|l|l|l|l|l|}
% 		\hline
% 		$G$&$m_1$&$m_2$&$m_3$&$m_4$&$K$\\
% 		\hline
% 		$g_1$&0&1&1&0&1\\ \hline
% 		$g_2$&1&2&0&1&1\\ \hline
% 		$g_3$&0&1&0&1&1\\ \hline
% 		$g_4$&1&2&1&0&2\\ \hline
% 		$g_5$&1&1&0&1&2\\ \hline
% 		$g_6$&1&1&1&2&2\\ \hline		
% 	\end{tabular}	
% 	\normalsize% возвращаем шрифт к нормальному
% \end{table}

